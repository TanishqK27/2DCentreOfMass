%! Author = tanishqkumar
%! Date = 10/04/2024

\documentclass{article}
\usepackage{amsmath}
\begin{document}

\section*{Mathematical Explanation for ShapePlotter}

\subsection*{Center of Mass Calculation}

The center of mass (centroid) for a simple polygon is calculated using the formula:

\[
C_x = \frac{1}{6A} \sum_{i=0}^{n-1} (x_i + x_{i+1})(x_i y_{i+1} - x_{i+1} y_i)
\]

\[
C_y = \frac{1}{6A} \sum_{i=0}^{n-1} (y_i + y_{i+1})(x_i y_{i+1} - x_{i+1} y_i)
\]

where $A$ is the area of the polygon, calculated as:

\[
A = \frac{1}{2} \sum_{i=0}^{n-1} (x_i y_{i+1} - x_{i+1} y_i)
\]

Here, $(x_0, y_0), \ldots, (x_{n-1}, y_{n-1})$ are the coordinates of the vertices of the polygon, and $(x_n, y_n) = (x_0, y_0)$ to close the polygon.

\subsection*{Area Calculation}

The area of the polygon is calculated using the shoelace formula:

\[
A = \frac{1}{2} \left| \sum_{i=0}^{n-1} (x_i y_{i+1} - x_{i+1} y_i) \right|
\]

This formula calculates the signed area, so the absolute value is taken to ensure the area is positive.

\subsection*{Rotation Calculation}

The rotation of a point $(x, y)$ around a pivot point $(p_x, p_y)$ by an angle $\theta$ is calculated as follows:

\[
\begin{bmatrix}
x' \\
y'
\end{bmatrix}
=
\begin{bmatrix}
\cos(\theta) & -\sin(\theta) \\
\sin(\theta) & \cos(\theta)
\end{bmatrix}
\begin{bmatrix}
x - p_x \\
y - p_y
\end{bmatrix}
+
\begin{bmatrix}
p_x \\
p_y
\end{bmatrix}
\]

This rotation matrix ensures that the shape rotates as a rigid body around the pinning point, maintaining its structure while changing orientation.

\end{document}